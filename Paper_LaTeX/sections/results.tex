\section{Results}

For the final layout, we choose to use the 30 most common characters in our corpus, which is the standard English alphabet, plus the addition of four special characters: the comma, period, hyphen, and apostrophe. This layout differs by 2 characters from the QWERTY layout, so to compare the final layout with QWERTY requires we only predict the time for strings in the corpus that can be typed on both layouts. Although the average typing speed was $47$ WPM, we set the target WPM to $\geq80$ as a litmus test of proficiency. This results in single-finger bistrokes having a greater impact and frequency having less impact. Optimizing for an above-average WPM like this is desirable if the goal is to raise the upper threshold of potential typing speed. Using the methodology in this paper, one could also aim to optimize for the average user, improving the mean speed. The resulting layout can be seen in Table \ref{fig:keyboard1}.
\begin{table}[h]
\caption{Generated Layout for Typing Speeds $\geq80$ WPM}
\begin{center}
\tiny
\renewcommand{\arraystretch}{1.5}
\begin{tabularx}{\columnwidth}{ | *{10}{>{\centering\arraybackslash}X|} }
\hline
\textbf{M} & \textbf{R} & \textbf{T} & \textbf{C} & \textbf{W} & \textbf{,} & \textbf{K} & \textbf{A} & \textbf{E} & \textbf{'} \\ \hline
\textbf{L} & \textbf{N} & \textbf{D} & \textbf{S} & \textbf{V} & \textbf{Y} & \textbf{U} & \textbf{O} & \textbf{I} & \textbf{G} \\ \hline
\textbf{H} & \textbf{X} & \textbf{Z} & \textbf{B} & \textbf{F} & \textbf{.} & \textbf{-} & \textbf{Q} & \textbf{J} & \textbf{P} \\ \hline
\multicolumn{10}{|c|}{\multirow{1}{*}{\textbf{SPACE}}} \\ \hline
\end{tabularx}
\end{center}
\label{fig:keyboard1}
\end{table}




\noindent The predicted time to type the iWeb corpus on QWERTY is 54,934,582 seconds, Dvorak is 52,565,249 seconds, a speed up of $~4\%$, and the layout produced in this paper is predicted to take 51,429,827 seconds, a speed up of $~6\%$. Our analysis revealed positional categories of bistroke that played a significant role in the prediction of typing speed. Namely, ALTs which are shown to be faster and SFBs which are shown to be slower. For QWERTY, $18.3\%$ of all bistrokes are ALTs and $5.7\%$ of them are SFBs, dvorak has $33.6\%$ ALTs and $2.8\%$ SFBs, and the layout produced in this paper has the best result of $33.6\%$ ALTs and $1.4\%$ SFBs. The fact that speed optimization yields only marginal improvements in typing efficiency is not unexpected but remains a valuable take away as it suggests prioritizing theoretical features over solely focusing on speed optimization may offer substantial benefits.

% colemak and dvorak here

\section{Future Work}
The overrepresentation of the QWERTY, AZERTY, and QWERTZ layouts in the current dataset significantly biases the model, as evidenced by the preference for the top row. This bias impairs the model's ability to generalize accurately. Although the proposed cost function is simple and generalizable, it fails to fully capture the complexities of human typing behavior. Future studies should prioritize the collection of data from a wider variety of keyboard layouts and typing speeds to mitigate this limitation. To facilitate this, an open-source tool, Kiakl, was developed to crowdsource data from alternative keyboard layouts. However, due to time constraints and the limited volume of data gathered, Kiakl was not used in this study. A more comprehensive dataset would enable the exploration of more sophisticated models, such as transformers or recurrent neural networks, offering the potential for a more nuanced and accurate evaluation of keyboard layouts.

The incorporation of previously excluded tristroke and bistroke features also warrants consideration. These features include lateral stretches, redirects, rolls, and scissors. A lateral stretch occurs when two adjacent fingers are pulled apart horizontally, complicating the articulation of a bistroke. These occur almost exclusively from keystrokes in the inner index columns. Scissors occur when one finger of a hand reaches to the top row while another finger on the same hand contracts to press a key on the bottom row. Redirects occur when a tristroke, typed with the same hand, changes direction, frequently resulting in typos. An example of this in the QWERTY layout is the tristroke "sad," as the "sa" is typed outward while "ad" is typed inward. A roll occurs when a tristroke is typed with the same hand and maintains a consistent direction, which may be faster to articulate. Modeling these movements may increase predictive accuracy offering opportunities for further optimization. 
% These movements, though subtle and nuanced, represent important articulation dynamics that can influence typing speed and efficiency, offering opportunities optimization.

Further improvements to the keyboard optimization process could involve refining the simulated annealing algorithm itself. Although geometric cooling was effective in this study, exploring alternative cooling schedules could improve performance. Modifying the cooling rate over time or experimenting with alternative patterns, such as polynomial or logarithmic decays, may enhance the algorithm's ability to find global optima. Finally, incorporating formal mathematical validation, such as estimating the lower bound of optimization, would help verify and guide future results.
% left hand right hand
% Integrating other metaheuristic algorithms, like genetic algorithms or particle swarm optimization, could provide valuable insights and alternative approaches for optimization.

% more data, kiakl, looking for ways of measuring comfort that are valid