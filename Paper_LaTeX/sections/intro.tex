

\section{Introduction}
% maybe rewrite, with the invention of the typewriter, yada yada
% The creation of text input methods tailored to English trace back to the era of typewriters. 
QWERTY is a keyboard layout first created for typewriters by Christopher Sholes in the early 1870s \citep{yasuoka2011prehistory}. Today, 150 years since its invention, QWERTY is the most common keyboard layout and, by extension, the most common means of human-computer interaction, extending even to mobile devices. However, what was innovative then has become a source of concern in the modern era, as the design of QWERTY, conceived with little regard for ergonomics, has been linked to user discomfort and repetitive strain injuries \citep{amell2000cumulative}.
% and was further developed and popularized with the success of the Remington No. 2 typewriter, which featured the layout

One of the earliest and most influential advocates for alternative keyboard layouts was August Dvorak, who, along with his collaborator William Dealey, championed a more scientific approach to keyboard design. Their methodology involved motion studies, analysis of high-frequency letters and letter combinations, and a deep understanding of hand physiology, culminating in the creation of the Dvorak Simplified Keyboard in 1932 \citep{hiraga}. Dvorak's efforts in the 1930s marked a significant shift toward a more scientific approach to keyboard design and, at the very least, planted the idea that a more efficient alternative to the QWERTY layout was possible.

Despite these early endeavors, the field has continued to struggle to quantify the precise impact of ergonomic factors on typing efficiency and user comfort. This has led to a continued reliance on theoretical models informed by qualitative observations rather than empirical models derived from robust quantitative data. From these observations, several standard practices have emerged, such as minimizing the distance between keys, arranging columns based on perceived finger dexterity, limiting the use of the same finger twice in succession, and considering character frequency \citep{light1993typewriter}. % For example, \citet{light1993typewriter} employed character frequency and predicted finger travel times from calculated distances as their optimization criterion.

Beyond quantifying the impact of ergonomic features, finding an optimal layout presents a significant computational challenge. For $n$ keys, there are $n!$ possible layouts, making the problem combinatorially complex. Even for the 26 letters of the English alphabet, this complexity renders brute-force solutions infeasible. This complexity places the problem among other well-known combinatorial problems, such as the Traveling Salesman Problem (TSP) and the Quadratic Assignment Problem (QAP). In the TSP, QAP, and keyboard layout optimization, the goal is to minimize a cost as it relates to distance: for the TSP, the distance is between cities; for the QAP, the distance is between facilities; and for keyboard optimization, the distance is between keys. The QAP, unlike the TSP, also accounts for flow, which measures the intensity of interactions (e.g., between related facilities)  -- analogous to character frequency in keyboard layout optimization. Given these similarities, researchers have applied a multitude of combinatorial optimization algorithms to address keyboard layout optimization, such as simulated annealing \citep{light1993typewriter}, ant colony optimization \citep{eggers2003optimization}, swarm optimization \citep{yin2011cyber}, and genetic algorithms \citep{liebrock2005proceedings}, among others.

This paper presents a novel, data-driven approach to keyboard layout optimization. We propose a quantitative objective function that evaluates keyboard layouts based on key placement, finger dexterity, and character frequency, informed by a comprehensive analysis of real-world typing data and a large English corpus. To optimize these layouts, we implement a custom version of the simulated annealing algorithm that dynamically adjusts the initial temperature and termination criterion. This approach allows for the generation of keyboard layouts tailored to specific typing speed ranges, corpora including other languages, and individual user needs.

Typing is a complex interaction between physical constraints, muscle memory, and cognitive processes \citep{hiraga}. An optimal design must not only account for these complex interactions, but also make decisions that, at times, may seem counterintuitive to balance them. This study focuses on optimizing one clear and measurable aspect of typing -- typing speed. While typing comfort is equally important, it remains a multifaceted and subjective concept that is difficult to quantify. By focusing our efforts on typing speed, we aim to establish a scientifically sound foundation. Comfort, on the other hand, may be where human artistry and subjective preferences come into play, reflecting a more individualized aspect of keyboard design. %that extends beyond strict optimization.
% This complexity makes keyboard optimization particularly challenging to define.

% , characterized by a "leveling" effect where typists unconsciously adjust their typing rhythm to smooth out variations in difficulty between consecutive keystrokes 